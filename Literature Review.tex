\documentclass[10pt,a4paper]{article}
\usepackage{zed-csp,graphicx,color}
\usepackage{cite}
\begin{document}
\begin{titlepage}
 \begin{figure}[h]
  \centerline{\small MAKERERE 
  \includegraphics[width=0.1\textwidth]{muklog} UNIVERSITY}
\end{figure}
\centerline{COLLEGE OF COMPUTING AND INFORMATIC SCIENCES}
\paragraph{•}
\centerline{DEPARTMENT OF COMPUTER SCIENCE\\}
\paragraph{•}

\centerline{COURSEWORK THREE: RESEARCH METHODOLOGY(BIT 2207)\\}
\paragraph{•}
\centerline{LECTURER: MR.ERNEST MWEBAZE}
\paragraph{•}
\centerline{A SYSTEMATIC LITERATURE REVIEW OF CLOUD COMPUTING IN EHEALTH\\}
\paragraph{•}
\centerline{COMPILED BY
 ARIHO IGNATIUS}
 \paragraph{•}
\centerline{STUDENT NUMBER : 216001851}\
\paragraph{•}
\centerline{REGISTRATION NUMBER:16/U/1811}
\paragraph{•}
%\maketitle
\end{titlepage}
\pagenumbering{roman}
\tableofcontents
\newpage
\pagenumbering{arabic}
\section{Introduction}
EHealth is defined as “the cost-effective and secure use of information and communications
technologies in support of health and health-related fields, including healthcare services, health surveillance, health education, knowledge and research”. The goal of eHealth is to improve the cooperation and coordination of healthcare, in order to improve the quality of care and reduce the cost of care at the same time.
Cloud computing is a new technology which has emerged in the last five years. \cite{T1} According to the definition by NIST, cloud computing is “a model can provide distributed, rapidly provisioned and configurable computing resources such as servers, storage, applications and networks. Because of the obvious scalability, flexibility and availability at low cost of cloud services, there is a rapid trend of adopting cloud computing among enterprises or health related areas in the last few years.

\section{Application of Cloud Computing} 
\cite{T2} High accessibility, availability and reliability make cloud computing a better solution for healthcare interoperability problems. Papers in this category mostly applied cloud technology for healthcare data sharing, processing and management, and can be categorized based on three types of cloud platforms, namely, public cloud, private cloud and hybrid cloud. For applications based on the private cloud, Bahga et al. presented an achievement of sematic interoperability between different kinds of healthcare data.

\section{Security or Privacy Control Mechanism}
Healthcare data require protection for high security and privacy. Access control, an effective
method to protect data, is widely used in many studies.\cite{T3} Liu et al. applied an identity-based encryption (IBE) system in access control of PHR, and this identity-based cryptography system can reduce the complexity of key management. Attribute based Encryption (ABE) is one of the most preferable encryption schemes used in cloud computing.

\section{Discussions}
The presented review shows that cloud computing technology could be applied in several areas of the eHealth domain.
The majority of studies introduced cloud computing technology as possible solutions for achieving eHealth interoperability. Although worldwide it is acknowledged that ICT technologies, such as cloud computing, can improve healthcare quality.

\section{Conclusion}
Research on applying cloud computing technology to eHealth is in its early stages; most researchers have presented ideas without real-world cases validation. The obvious features of cloud computing technology provide more reasons to adopt cloud computing in sharing and managing health information. 
\bibliographystyle{IEEEtran}
\bibliography{data}
\end{document}